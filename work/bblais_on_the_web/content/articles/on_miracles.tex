title: On Miracles subtitle: Hume was right all along date: 2015-05-13

After my \href{http://web.bryant.edu/~bblais/misquoting-hume.html}{post
about the misquoting Hume} and the miraculous, I was recommended to read
the Stanford Encyclopedia of Philosophy
\href{http://plato.stanford.edu/entries/miracles/}{article on miracles}
where it supposedly explains why ``Hume was toast'' before any
misquoting of him. In this post, I hope to sketch the relevant parts of
the encyclopedia article, and demonstrate how it does not establish any
significant deficiency of Hume. I use the same headings as the
\href{http://plato.stanford.edu/entries/miracles/}{article on miracles}
to make it easier to follow.

\section{Concepts and Definitions}\label{concepts-and-definitions}

The article begins by discussing one of Hume's definitions of a miracle
as ``a violation of the laws of nature''. From what I can tell, their
main critique is they don't like the connotations of the word ``law'', a
perspective I share - it is a bit loose terminology, with too many
alternate meanings to be the foundation of a well-defined argument.
Their revised definition is the following:

\begin{quote}
A miracle is an event that exceeds the productive power of nature
\end{quote}

Perhaps this is the scientist in me, and why I am not a philosopher, but
I don't see a striking difference between these two definitions in at
least how they are used. So it seems reasonable to adopt this as a good
working definition.

They go on to clarify a subset of miracle,

\begin{quote}
a religiously significant miracle is a detectable miracle that has a
supernatural cause.
\end{quote}

This clarification is to deal with the following problem, and I'd agree
with at least the sentiment.

\begin{quote}
An insignificant shift in a few grains of sand in the lonesome desert
might, if it exceeded the productive powers of nature, qualify as a
miracle in some thin sense, but it would manifestly lack religious
significance and count not be used as the fulcrum for any interesting
argument.
\end{quote}

I am not sure how, in practice, one would be able to determine a
``supernatural cause'', let alone establish how an event could be beyond
the ``productive power of nature'' without committing a fallacy of
\emph{argument from ignorance}, but let's leave that for now.

\section{Arguments for Miracle
Claims}\label{arguments-for-miracle-claims}

This section starts with a quick list of the types of evidence and
arguments made for miracles.

\begin{quote}
Many arguments for miracles adduce the testimony of sincere and able
eyewitnesses as the key piece of evidence on which the force of the
argument depends. But other factors are also cited in favor of miracle
claims: the existence of commemorative ceremonies from earliest times,
for example, or the transformation of the eyewitnesses from fearful
cowards into defiant proclaimers of the resurrection, or the conversion
of St.~Paul, or the growth of the early church under extremely adverse
conditions and without any of the normal conditions of success such as
wealth, patronage, or the use of force. These considerations are often
used jointly in a cumulative argument. It is therefore difficult to
isolate a single canonical argument for most miracle claims. The various
arguments must be handled on a case-by-case basis.
\end{quote}

All of these pieces of so-called evidence are the worst kind of
evidence, for which there are countless examples of the same, or similar
evidence use to shore up the claims of other (presumably false) beliefs.
You can think ``Mormonism'' or ``Alien Abductions'' for nearly every
point listed.

They then outline two types of inductive arguments:

\begin{quote}
\begin{enumerate}
\def\labelenumi{\arabic{enumi}.}
\itemsep1pt\parskip0pt\parsep0pt
\item
  the conclusion (in this case that the miracle in question has actually
  occurred) is probable to some specific degree, or at least more
  probable than not
\item
  the conclusion is more probable given the evidence presented than it
  is considered independently of that evidence
\end{enumerate}
\end{quote}

Point (1) is just either specifying either
\(P({\rm miracle}|{\rm data})\) directly or establishing only that
\(P({\rm miracle}|{\rm data})>0.5\). Point (2) is
\(P({\rm miracle}|{\rm data})>P({\rm miracle})\). Point (2) is nearly
useless. For example, you could have

\begin{eqnarray*}
P({\rm miracle})&=&0.00001 \\
P({\rm miracle}|{\rm data})&=&0.001
\end{eqnarray*}

and still have a seriously unlikely hypothesis, even given a factor of
100 increase in probability of a miracle given the data. Thus the
\emph{only} thing that matters must be the actual value of
\(P({\rm miracle}|{\rm data})\).

One such argument for miracles specifies the type of evidence needed to
make one confident that one is talking about a miracle. The article
calls this a ``criteriological'' argument, but all of the arguments
dealt with are probabilistic. What are the criteria, for example? This
one is from Charles Leslie:

\begin{quote}
\begin{enumerate}
\def\labelenumi{\arabic{enumi}.}
\itemsep1pt\parskip0pt\parsep0pt
\item
  That the matters of fact be such, as that men's outward senses, their
  eyes and ears, may be judges of it.
\item
  That it be done publicly in the face of the world.
\item
  That not only public monuments be kept up in memory of it, but some
  outward actions to be performed.
\item
  That such monuments, and such actions or observances, be instituted,
  and do commence from the time that the matter of fact was done.
\end{enumerate}
\end{quote}

One can easily site both the golden plates of Joseph Smith, and also the
events surrounding Roswell, that satisfy all of these. Clearly, there is
an issue with them.

Another common argument is called the ``minimal facts'' approach. The
best summary, and take-down of this argument is on
\href{https://adversusapologetica.wordpress.com/2013/06/29/knocking-out-the-pillars-of-the-minimal-facts-apologetic/}{Matthew
Ferguson's blog}. One essential missing part of the minimal facts
approach is that it only includes \emph{likelihoods} and not
\emph{priors}, and thus fails a basic probabilistic analysis.

\subsection{Probabilistic arguments}\label{probabilistic-arguments}

The first form here deals with \emph{testimony}, with the following
assumptions and conventions:

\begin{enumerate}
\def\labelenumi{\arabic{enumi}.}
\itemsep1pt\parskip0pt\parsep0pt
\item
  \(T_i\equiv\) the proposition ``Witness \(i\) testifies to \(M\)''
\item
  \(P(T_i,T_j) = P(T_i)\times P(T_j)\): independence
\item
  \(P(T_i|M)=P(T_j|M)\) for all \(i\) and \(j\): all testimony is
  equally informative
\end{enumerate}

We then easily derive: \[
\frac{P(M|T_1,T_2,\cdots,T_n)}{P(\sim\!M|T_1,T_2,\cdots,T_n)} = \left(\frac{P(T_1|M)}{P(T_1|\sim\!M)}\right)^n \times \frac{P(M)}{P(\sim\!M)}
\]

The article then spins this in a positive way toward miracles:
\textgreater{}{[}I{]}f independent witnesses can be found, who speak
truth more frequently than falsehood, \emph{it is ALWAYS possible to
assign a number of independent witnesses, the improbability of the
falsehood of whose concurring testimony shall be greater than the
improbability of the alleged miracle.} (Babbage 1837: 202, emphasis
original; cf.~Holder 1998 and Earman 2000)

However, comparing with Hume, it becomes obvious why this spin fails:
\textgreater{} When anyone tells me, that he saw a dead man restored to
life, I immediately consider with myself, whether it be more probable,
that this person should either deceive or be deceived, or that the fact,
which he relates, should have really happened. I weigh the one miracle
against the other; and according to the superiority, which I discover, I
pronounce my decision, and always reject the greater miracle. If the
falsehood of the testimony would be more miraculous, than the event
which he relates; then, and not till then, can he pretend to command my
belief or opinion. (Hume)

The first quote implies that the terms \(P(T_1|M)\) and
\(P(T_1|\sim\!M)\) refer to speaking truth vs falsehoods (i.e.~lying),
as opposed to speaking correctly vs being mistaken. In the latter, it is
very easy to see why, for certain types of extraordinary events, we
would expect fallible observers to have \(P(T_1|\sim\!M)>P(T_1|M)\) and
further that even \emph{if} witnesses were in general slightly more
reliable than not, we can't expect the observations to be
\emph{independent} in general. In the specific case of the (anonymous)
Gospel writers, there is strong evidence of \emph{dependence} in the
accounts to make this entire calculation (except in its gross
qualitative features) irrelevant.

\section{Arguments against miracles}\label{arguments-against-miracles}

Quoting Hume again,

\begin{quote}
The plain consequence is (and it is a general maxim worthy of our
attention), ``That no testimony is sufficient to establish a miracle,
unless the testimony be of such a kind, that its falsehood would be more
miraculous, than the fact, which it endeavors to establish: And even in
that case, there is a mutual destruction of arguments, and the superior
only gives us an assurance suitable to that degree of force, which
remains, after deducting the inferior.''
\end{quote}

This is correct, and is a direct statement of Bayesian reasoning \[
\frac{P(M|E)}{P(\sim\!M|E)} =\frac{P(E|M)}{P(E|\sim\!M)} \times\frac{P(M)}{P(\sim\!M)}
\] where we can use the approximations \(P(E|M)\approx 1\) and
\(P(\sim\!M)\approx 1\) and achieve \[
\frac{P(M|E)}{P(\sim\!M|E)} \approx \frac{P(M)}{P(E|\sim\!M)}
\]

The \href{http://plato.stanford.edu/entries/miracles/}{article on
miracles} continues to try to map this to a philosophical structure
(needlessly, I'd say), with the following ``simple version'' of the
argument:

\begin{quote}
A very simple version of the argument, leaving out the comparison to the
laws of nature and focusing on the alleged infirmities of testimony, can
be laid out deductively (following Whately, in Paley 1859: 33):

\begin{enumerate}
\def\labelenumi{\arabic{enumi})}
\item
  Testimony is a kind of evidence very likely to be false.
\item
  The evidence for the Christian miracles is testimony.
\end{enumerate}

Therefore,

\begin{enumerate}
\def\labelenumi{\arabic{enumi})}
\setcounter{enumi}{2}
\itemsep1pt\parskip0pt\parsep0pt
\item
  The evidence for the Christian miracles is likely to be false.
\end{enumerate}

This is, however, much too crude an argument to carry any weight, since
it turns on a simple ambiguity between all testimony and some testimony.
Whately offers an amusing parody that makes the fallacy obvious: Some
books are mere trash; Hume's Works are {[}some{]} books; therefore, etc.
\end{quote}

It's remarkable that such a silly parallel is seriously made. The
structure isn't really parallel at all, so let's make it explicit:

\begin{enumerate}
\def\labelenumi{\arabic{enumi}.}
\itemsep1pt\parskip0pt\parsep0pt
\item
  Books are likely to be trash. (in other words, most books are trash)
\item
  Hume wrote some books
\item
  therefore, Hume's books are likely to be trash.
\end{enumerate}

This is a correct argument, given the premises. If all we knew was that
``some guy named Hume'' wrote ``some book'', then with all probability
(if premise 1. is correct) that book would be trash. The issue is that,
unconsciously, we are inserting extra information - Hume was a famous
philosopher, he had a particular education, etc\ldots{} With this extra
information, we would have a hard time supporting a similar premise as
1. above.

The fact that this is so trivially dispensed with makes one wonder - why
would anyone be convinced by this? Why couldn't the author of the
article see it? It smacks of grasping at straws to try to dispel Hume's
main arguments.

The article continues with some odd re-phrasings of Hume, where the
mathematics is just the single line above. I don't understand all the
work. A strange one is then critiqued with an even stranger statement:
\textgreater{}The presumptive case against the resurrection from
universal testimony would be as strong as Hume supposes only if,
\emph{per impossible}, all mankind throughout all ages had been watching
the tomb of Jesus on the morning of the third day and testified that
nothing occurred. Even aside from the problems of time travel, there is
not a \emph{single piece} of direct testimonial evidence to Jesus'
non-resurrection.

Does anyone seriously think that the case against a claim always (or
even usually) takes the form of direct testimony against that claim?
Where is the testimony that Zeus didn't exist? Anyone who can explain
this odd line of reasoning, please chime in.

\section{Particular Arguments}\label{particular-arguments}

According to the article, Hume lists a set of conditions needed to make
testimony carry maximum weight:

\begin{quote}
{[}T{]}here is not to be found, in all history, any miracle attested by
a sufficient number of men, of such unquestioned good sense, education,
and learning, as to secure us against all delusion in themselves; of
such undoubted integrity, as to place them beyond all suspicion of any
design to deceive others; of such credit and reputation in the eyes of
mankind, as to have a great deal to lose in case of their being detected
in any falsehood; and at the same time attesting facts, performed in
such a public manner, and in so celebrated a part of the world, as to
render the detection unavoidable: All which circumstances are requisite
to give us a full assurance in the testimony of men. (Hume 1748/2000:
88)
\end{quote}

Essentially, he is saying that the methods of science have never
confirmed a miracle. The methods of science help ``secure us against all
delusion in themselves'', remove ``suspicion of any design to deceive
others'', with processes ``performed in a public manner'' that ``render
the detection unavoidable''.

It is criticized by noting that some of these conditions can cut the
other way, such as the condition of ``credit and reputation'',

\begin{quote}
It might have been said with some shew of plausibility, that such
persons by their knowledge and abilities, their reputation and interest,
might have it in their power to countenance and propagate an imposture
among the people, and give it some credit in the world. (Leland 1755:
90--91; cf.~Beckett 1883: 29--37)
\end{quote}

This is, essentially, pointing out fallacy of authority - a good
critique. Science, by its processes, attempts to avoid that as well. Of
course, Hume predates modern science, so I think we can forgive him some
sloppiness or poor choice of terminology.

Hume continues to suggest that the religious context of the miracle
claime makes the telling of the miracle story even more likely. This
would increase the probability of obtaining the testimony even if no
miracle happened - \(P(E|\sim\!M)\) increases - making the probability
of a miracle go down. The criticism here? The effect could happen in the
other direction:

\begin{quote}
But as George Campbell points out (1762/1839: 48--49), this
consideration cuts both ways; the religious nature of the claim may also
operate to make it less readily received:

\begin{quote}
{[}T{]}he prejudice resulting from the religious affection, may just as
readily obstruct as promote our faith in a religious miracle. What
things in nature are more contrary, than one religion is to another
religion? They are just as contrary as light and darkness, truth and
error. The affections with which they are contemplated by the same
person, are just as opposite as desire and aversion, love and hatred.
The same religious zeal which gives the mind of a Christian a propensity
to the belief of a miracle in support of Christianity, will inspire him
with an aversion from the belief of a miracle in support of
Mahometanism. The same principle which will make him acquiesce in
evidence less than sufficient in one case, will make him require
evidence more than sufficient in the other\ldots{}.
\end{quote}
\end{quote}

I disagree quite strongly with this line of thinking. One of the big
problems with pseudoscience is that it promotes poor thinking in other
domains. Someone who believes in miracles will not find it hard to
believe that the miracle claims of other religions are at least
possible. If you believe in unseen agents, then to move from
Christianity to New Age to Scientology isn't that large of a stretch.
Often, when ones religion is undermined, the typical response is to
switch to another religion! Thus, they are not as opposite as ``light
and darkness''. Poor thinking is poor thinking, regardless of the
context.

\subsection{Argument from Parity}\label{argument-from-parity}

Hume brings up miracles in other religions. In a fit of special
pleading, the \href{http://plato.stanford.edu/entries/miracles/}{article
on miracles} retorts,

\begin{quote}
All attempts to draw an evidential parallel between the miracles of the
New Testament and the miracle stories of later ecclesiastical history
are therefore dubious. There are simply more resources for explaining
how the ecclesiastical stories, which were promoted to an established
and favorably disposed audience, could have arisen and been believed
without there being any truth to the reports.
\end{quote}

The argument is quite simple - if there are known cases of miracle
claims where no miracle actually occurred, that increase
\(P(E|\sim\!M)\), making the probability of a miracle go down given
testimony. It doesn't matter whether you have good reasons to believe
there was no miracle for these cases - it undermines testimony of
miracles in general.

\section{In conclusion}\label{in-conclusion}

So, as far as I can tell, there is no substantive critique to Hume's
statements about miracles. He lacks the rigor of the mathematics of
probability, but his wording is so straightforwardly translated to it
that I find it difficult to see what the problem is. I also found it
ironic that the entire article, which has been pro-miracle the entire
time, ends with this:

\begin{quote}
For the evidence for a miracle claim, being public and empirical, is
never strictly demonstrative, either as to the fact of the event or as
to the supernatural cause of the event. It remains possible, though the
facts in the case may in principle render it wildly improbable, that the
testifier is either a deceiver or himself deceived; and so long as those
possibilities exist, there will be logical space for other forms of
evidence to bear on the conclusion. Arguments about miracles therefore
take their place as one piece---a fascinating piece---in a larger and
more important puzzle.
\end{quote}

This is pretty much exactly what Hume was saying. Given that there is
always a non-zero probability of the testifier either lying or being
mistaken, one has to establish the evidence for a miracle strong enough
to overcome both the negligibly small prior probability and this
non-zero probability of the testimony being wrong. Since mistakes are a
common human trait, and distortions are also common on testimony,
evidence for miracles according to probability theory, Hume, and all
rational thought have always been found lacking.

Of course, if you could demonstrate it otherwise, please let me know!
I'd love to believe in miracles. I just have never seen anything even
remotely convincing.

For some other items I've written about miracles, see:

\begin{enumerate}
\def\labelenumi{\arabic{enumi}.}
\itemsep1pt\parskip0pt\parsep0pt
\item
  \url{http://web.bryant.edu/~bblais/do-healing-miracles-happen.html}
\item
  \url{http://web.bryant.edu/~bblais/a-little-about-miracles.html}
\item
  \url{http://web.bryant.edu/~bblais/unbelievable-project-miracles-and-healing-is-it-evidence-for-the-truth-of-christianity.html}
\end{enumerate}
